Маршрут начинается прямо от хижины \geoLighthouse и строго от этой
точки уходит вверх по каменной россыпи. Подъём по тропе без отворотов
идет до начала второстепенного гребня пика Учитель. Там есть
\geopoint{42.5387431}{74.5375909}{довольно большая ровная площадка для
  отдыха}.

Дальше тропа уходит направо, на этот гребень. Подъем по нему
однозначен, идёт серпантином. Хороший ориентир~--- \geoUchitelUshi.
Уши обходятся либо слева по снегу, либо справа по скалам. Дальше точно
так же пешком до выхода на основной гребень. По основному гребню
двигаться направо 10--30 минут до вершины \geoPeakUchitel.

Спуск по пути подъема. Можно продолжить движение по гребню в сторону
вершины Байчечекей и таким образом пройти траверс 2Б к.с.

Маршрут проходится полностью пешком. Если вам где-то пришлось начать
держаться за скалы руками~--- подумайте, скорее всего вы ушли не туда.
