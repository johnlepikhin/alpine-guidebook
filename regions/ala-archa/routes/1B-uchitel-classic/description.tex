\routeSection{R0-R1, \length{800}}{ Маршрут начинается прямо от хижины
\geoLighthouse{} и строго от этой точки уходит вверх по каменной
россыпи. Подъём по тропе без отворотов идет до начала
второстепенного гребня пика Учитель. Там есть
\geopoint{42.5387431}{74.5375909}{довольно большая ровная площадка
для отдыха}. }

\routeSection{R1-R2, \length{900}}{Тропа уходит направо, на подходящий
  справа гребень. Подъем по нему однозначен, идёт серпантином. Хороший
  ориентир~--- \geoUchitelUshi{}.}

\routeSection{R2-R3, \length{200}}{Уши обходятся либо слева по снегу,
  либо справа по скалам. Дальше точно так же пешком до выхода на
  основной гребень.}

\routeSection{R3-R4, \length{300}}{По основному гребню двигаться
  направо 10--30 минут до вершины \geoPeakUchitel{}.}

Можно продолжить движение по гребню в сторону вершины Байчечекей и
таким образом пройти траверс 2Б к.с.

\routeSection{R4-R0}{Спуск по пути подъема.}

Маршрут проходится полностью пешком. Если вам где-то пришлось начать
держаться за скалы руками~--- подумайте, скорее всего вы ушли не туда.
