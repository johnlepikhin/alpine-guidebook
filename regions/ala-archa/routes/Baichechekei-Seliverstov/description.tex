\routeSection{R0-R1, \length{1200}}{От хижины \geoLighthouse{} тропа
уходит вверх на морены ледника Учитель. По ним двигаться ближе к
левому борту (под ходу движения) около часа до появления на левом
борту \geopoint{42.5280865}{74.544038}{довольно высоких скал}.}

\routeSection{R1-R2, \length{650}}{Немного недоходя скал начинается
подъём вверх. До входа в ущелье подниматься около 30 минут, далее по
ущелью ещё около 30 минут. Необходимо подняться примерно на веревку
выше ледового кулуара «Речки», до
\geopoint{42.53058314088}{74.55011128306}{начала льда} в основном
кулуаре.}

\routeSection{R2-R3, \length{240}}{Лезть вдоль правого борта,
постепенно забирая налево, в
\geopoint{42.53063716783}{74.5541706386}{узкое ледовое горлышко}, до
которого от старта маршрута около 4 веревок.}

\routeSection{R3-R4, \length{170}}{После прохождения горлышка маршрут
поворачивает направо. Необходимо лезть по льду до самого верха, не
поддаваться соблазну свернуть в более близкие кулуарчики на гребне.}

\routeSection{R4-R5, \length{140}}{\routeKey{Финиш маршрута, он же
ключ, становится видно после ледового горлышка: это более крутая
стеночка льда, длиной около 2.5 веревок, выводящая на
\geopoint{42.52951064326}{74.55579369317}{выраженную перемычку}}.
Вылезать на перемычку лучше в левой части, там удобная площадка для
нескольких человек, на которой можно собрать снаряжение.}

Спуск пешком по осыпному кулуару на противоположную сторону. Он
выводит на ледник Учитель.
