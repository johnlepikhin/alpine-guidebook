От хижины \geoLighthouse{} подняться по \geoKrokodil[морене «Крокодил»] до самого верха морены.

\photo{\routeDiskPath}{1}{\routeTitle~--- нитка маршрута}

\routeSection{R0-R1, \Gls{uiaaIII}}{Можно начать лезть по разрушенным скалам прямо вверх от ночёвок, а можно зайти в кулуар между пиком
Рацека и безымянным жандармом и оттуда по явно выраженному ответвлению по простым скалам выйти под большой скальный жандарм на гребне.
Жандарм обходится слева по троечным скалам.}

\routeSection{R1-R2, \Gls{uiaaIII}}{Затем маршрут выходит тоже по троечным скалам на небольшую перемычку в гребне и с неё начинается
ключевой участок.}

\routeKey{R2-R3, \length{20}, \Gls{uiaaIV}}{Лезется сначала несколько метров по канту и затем слева от канта гребня. Протяжённость участка
порядка двадцати метров.}

\routeSection{R4-R5, \length{100}, \Gls{uiaaIII}}{Маршрут выходит на простые двоечно/троечные скалы, по которым через две верёвки выход на
\geoPeakRacek{}.}

Спуск с вершины возможен по петлям по маршруту 2А до перемычки, либо по анкерным станциям на цепях, первая из которых находится примерно в
25 метрах ниже вершины.

Чтобы выйти на неё, надо пройти вниз по гребню в сторону 3А метров двадцать и оттуда можно пешком слезть на полку, на которой слева на
большом камне будет дюльферная станция. Следующая станция находится прямо на перегибе стенки над кулуаром. Необходима осторожность при
подходе. С этой станции один дюльфер до спускового кулуара.

Этот маршрут нельзя назвать полноценной 3А так как он слишком короткий. Но это отличный вариант для скальных занятий в боевой обстановке.
