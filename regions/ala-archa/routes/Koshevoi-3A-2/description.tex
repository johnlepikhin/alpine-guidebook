\routeSection{R0-R1, \length{700}}{За кухней хижины \geoLighthouse{}
начинается тропа сразу вверх, по ней подняться на 50 метров выше
последних скал слева.}

\routeSection{R1-R2, \length{550}}{Подняться на морену налево и
продолжить движение в том же направлении. Есть немного туров, но в
темноте их обычно не видно. Пересечь первую долинку, здесь лучше
двигаться по дуге без потери высоты. Пересечь вторую, довольно
обширную долину. На курумник справа смысла лезть нет, лучше обойти на
10 метров ниже. Подняться на гребешок 20 метров, примерно здесь
находится большой \geoStartOfKoshevoy[тур]. Но тропу в цирк
\geoPeakKoshevoy{} с этого гребня обычно видно и без тура.}

\routeSection{R2-R3, \length{500}}{Траверсировать склон по тропе с
постепенным спуском в цирк, к большим камням.}

\routeSection{R3-R4, \length{600}, \Gls{uiaaI}}{Квест по поиску туров
с записками начинается \geopoint{42.5457426}{74.5392475}{по светлой
сыпухе}, слева от башен. При проходе траверса в цирк этот кулуар
находится чуть левее направления движения. Совсем налево уходить
смысла нет, там туров нет и больше придется прыгать по развальне
гребня. Лучше сразу подниматься в кулуар непосредственно левее самой
левой башни.}

\routeSection{R4-R5, \length{300}, \Gls{uiaaV}}{Дальше идём по гребню,
ищем записки. В основном пешком в связке. По пути есть дюльфер 4 метра
(свободным лазанием зимой спуск затруднен), пара мест с IV лазанием по
паре метров. Подъем на пару башен требует простого пятерочного
лазания.}

\routeSection{R5-R6, \length{60}, \Gls{uiaaII}}{Центральную башню (сам
пик) обойти по правому борту по ходу движения. После нее в седловине
большой \geopoint{42.5470774253}{74.54563752883}{тур 1.5 метра}. От
него можно спуститься обратно в цирк, либо продолжить движение и
пройти последнюю башню до финишного кулуара.}

Спуск по одному из двух предложенных кулуаров в цирк, далее
возвращение в лагерь по пути подъема.
