От хижины \geoLighthouse{} подход 1.5-2.5 часа.

\routeSection{R0-R1, \length{50}}{Первая верёвка от берга как правило снег на льду. Бывает доска, неплохо подстраховаться.}

Далее развилка. Есть три варианта начала.

\begin{itemize}

\item Прямо вверх в кар. \routeSection{R1-R2, \length{150}}{Вас ждёт сначала короткая, но крутая и сыпучая стенка и потом ещё две с
половиной верёвки не сложного, но по снегам неприятного лазания по разрушенным скалам. Есть кусочки микста. Эти три верёвки выведут на
основную ледовую нитку. Нужны будут якоря и стоппера.}

\item Уйти налево по льду. \routeSection{R1-R2, \length{200}}{Уходим налево по льду и через верёвку, в узком скальном месте сворачиваем
направо вверх через скалы. Получаем метров двадцать интересного микста~--- драйтулинга и выходим на основную нитку верёвкой ниже, чем
центральный вариант. Можно обойтись без скального железа.}

\item Налево вверх до упора. \routeSection{R1-R2, \length{200}}{Уходим по льду налево вверх до упора и потом траверсируем по льду до
основной нитки, соединяясь с вариантом номер два.}

\end{itemize}

Варианты два и три через верёвку льда выходят на вариант номер один.

\routeSection{R2-R3}{Далее на варианте номер один, через верёвку, подходим под ключевой участок. На самом деле и подход достаточно крут
около 55ти градусов.}

\routeKey{R3-R4, \length{75}}{Ключ, участок длиной полторы верёвки крутизной градусов до семидесяти с узким горлышком, где надо аккуратно
долбить инструментами и смотреть по сторонам, так как на скалах есть ступеньки под ноги.

После выхода из горлышка логичное место под станцию в защищённом месте перед финишной верёвкой.}

\routeSection{R5-R6, \length{50}}{Первая половина финишной верёвки лёд градусов 55, переходящий в участок микста, выводящего на перемычку.
Финиш!}

Спуск с перемычки пешком на подушку Короны. Лучше в кошках, так как может быть жёстко.


Отличный тренировочный маршрут, особенно перед маршрутом Лоу на Свободную Корею, поскольку ключ по сложности и протяжённости такой же, как
нижний взлёт на Лоу. Плюс, при желании можно потренировать микст и драйтулинг. При этом маршрут не длинный, восемь верёвок и физически
вполне сносный.
