Классическая ледовая тройка района. Подход под маршрут 2-3 часа.
Надеть кошки и развесить снаряжение рекомендуется у
\geopoint{42.5154555}{74.5465833}{}{основания ледника}{основание ледника}.

\routeSection{R0-R1, \length{200}}{Маршрут начинается после берга. Подход под берг
пешком, часто тропежка. Берг как правило проходится без проблем.
Первые 4 верёвки лёд 50 градусов.}

\routeSection{R1-R2, \length{20}}{\routeKey{Ключ 60 градусов, около 20
метров. Проходить его в левой части ледника.}}

\routeSection{R2-R3, \length{170}}{Далее ещё 3-4 верёвки 50-градусного
льда до выполаживания. Про нему надо траверсировать под правые по ходу
движения скалы. Финиш комфортный, но иногда надо преодолеть пару
метров пологого зафирнованного наддува. Станция на большой балде.}

Спуск на другую сторону по пешему осыпному склону практически прямо к
\geoKoronaHut[Коронской хижине].
