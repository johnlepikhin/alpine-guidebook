От хижины \geoLighthouse{} подход 1.5-2.5 часа.

\photo{\routeDiskPath}{1}{\routeTitle~--- нитка маршрута}

На подходе важно вовремя свернуть направо в направлении языка ледника Учитель и не уйти выше под Байчечекей. На своротке с натоптанной тропы
под Байчечекей есть два больших, двойных тура. Также ориентиром может служить первая скальная гряда над Учительским ледником. Она находится
немного выше тропы.

Дальше, в зависимости от состояния снега можно либо выбраться на срединную морену и идти по ней, либо спуститься на ледник Учитель и
продолжить путь по нему, а через морену перебраться почти напротив маршрута.

При подходе под склон от последних камней необходимо быть внимательными, поскольку возможны опасные трещины.

\routeSection{R0-R1}{Бергшрунд обычно нормально преодолевается непосредственно под началом ледового кулуара справа от бастиона. По кулуару,
  лёд средней крутизны, надо вылезти налево на покатую осыпную полку, по которой выти налево до ледового ручейка, уходящего на гребень. В
  нижней части возможно стаивание льда и образование сложного для прохождения скально- микстового участка.}

\routeSection{R1-R2}{Далее маршрут проходит по достаточно крутому, узкому ледовому ручейку. При подходе к гребню надо быть внимательным,
  поскольку в непосредственной близости может возникнуть проблема с докапыванием до льда. Поэтому финишную станцию лучше организовать
  загодя, перед началом заснеженного участка.}

Спуск в сторону ледника Аксай.

Техничный, непростой ледово-микстовый маршрут, на который лучше выходить уже имея хороший навык лазания по скалам в кошках, микстового
лазания и драйтулинга.
