От \geoKoronaHut{коронской Хижины} подход час-полтора.

\photo{\routeDiskPath}{1}{\routeTitle~--- нитка маршрута}

\routeSection{R0-R1, \length{400}}{Первая часть маршрута~--- ледовый склон средней крутизны порядка восьми верёвок.}

\photo{\routeDiskPath}{2}{\routeTitle~--- скальная часть}

\routeSection{R1-R2, \length{50}, \Gls{uiaaV}}{Первая верёвка со льда неприятный микст с плохими возможностями для организации страховки.
  Сначала лезем по углу, потом забираем налево на снежный гребешок. По нему подходим под следующую стенку.}

\routeSection{R2-R3, \length{50}, \Gls{uiaaV}}{По стенке прямо вверх на полку, залитую льдом. Слева будет ориентир родной крюк с карабином.
  С полки лезем прямо вверх и выходим на следующую покатую полку с которой рельеф уводит чуть левее. Есть неприятное место с глушняком.}

\routeSection{R3-R4, \length{50}, \Gls{uiaaV}}{Третья верёвка заканчивается на развилке. Прямо заманивает залитый льдом желоб с родным
  железом, но лучше вылезти влево на крутые, но чистые скалы.}

\routeSection{R5-R6, \length{50}, \Gls{uiaaV}}{По скалам прямо вверх вылазим на гребень.}

Спуск в сторону первой башни. Дюльферяем с гребня вниз и дальше двигаемся под гребнем до выхода на спусковую петлю под первой башней. С неё
два дюльфера по гребню и третий в кулуар между Байляном и Космонавтами.

Дальше по кулуару. Есть родная петля на скальном острове чуть ниже перемычки. Потом своё. Потом надо смотреть направо на скалы. Есть родные
петли практически до самого низа.

Сложная, не клеточная пятёрка А с интересным микстовым лазанием.
