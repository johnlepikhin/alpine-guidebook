Подход под маршрут 2.5-3.5 часа. Маршрут идёт на 100-200 метров правее
основной линни маршррута Плотникова. Развесить железо рекомендуется у
\geopoint{42.5133135}{74.5543399}{}{основания ледника}{основание ледника}.

\routeSection{R0-R1, \length{100}}{На тропежку под берг надо заложить
20-40 минут времени, в зависимости от состояния снега. Берг не всегда
проходится просто, иногда надо поискать место оптимального пролаза.}

\routeSection{R1-R2, \length{500}}{При движении по маршруту надо
ориентироваться на хорошо видный на первых веревках большой скальный
\geopoint{42.5106817}{74.5545713}{}{зуб}{зуб}. До последних веревок маршрут
равномерный.}

\routeSection{R2-R3, \length{120}}{\routeKey{Ключом можно назвать
последние две веревки, где слева подходит слабо выраженный ледовый
гребешок, лед становится круче и почти до самого выхода на перемычку
приходится двигаться по диагонали. Иногда перед выходом на перемычку
возможен фирн. На этот случай лучше держать поблизости инструменты в
дополнение к фифм.}}

С перемычки приспуститься пешком 10 метров в ледовую мульду, из нее
подняться на ледовые подушки коронского ледника, пуск до
\geoKoronaHut[Коронской хижины] по закрытому леднику. Настоятельно
рекомендуется движение в связках, есть трещины! Развязаться можно
перед самым последним спуском.
