\routeSection{R0-R1, \length{2300}, пешком}{За кухней хижины
\geoLighthouse{} начинается тропа сразу вверх, по ней подняться на
\geoPeakUchitel{} по маршруту 1Б.}

\routeSection{R1-R2, \length{1000}, до \Gls{uiaaIII}}{От
вершины Учителя до \geoBaichechekey{} маршрут первое время маршрут
идёт прямо по гребню, но затем надо приспуститься на левый склон (по
ходу движения). В основном линия проходит не ниже 10-20 метров от
гребня. При большом количестве снега (лето) движение почти везде
одновременно в связке. Иногда простые дюльферы не больше 5 метров,
везде можно спуститься свободным лазанием. Быть внимательным при
прохождении лавиноопасных участков.}

\routeSection{R2-R3, \length{1500}, пешком}{Вершину обойти траверсом
слева, до подходящего слева перевального гребня.}

\routeSection{R3-R4, \length{120}, до \Gls{uiaaIII}}{По гребню подъём
до развальни одну верёвку, далее по ней простое лазание сложностью
II-III до вершины, ещё 1.5 верёвки.}

Спуск по этому же гребню на седловину, до
\geopoint{42.5288208}{74.5594939}{}{небольшого скальника}{небольшой
скальник}. Далее вниз направо, на ледник Учитель. Всё пешком.
