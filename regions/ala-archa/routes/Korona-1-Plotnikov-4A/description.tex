Самая популярная ледовая 4А района. Подход под маршрут 2.5-3.5 часа.
Развесить железо рекомендуется у
\geopoint{42.5133135}{74.5543399}{}{основания ледника}{основание лединка}.

\routeSection{R0-R1, \length{100}}{На тропежку под берг надо заложить
20-40 минут времени, в зависимости от состояния снега. Берг не всегда
проходится просто, иногда надо пройти влево-вправо вдоль него для
поиска оптимального места.}

\routeSection{R1-R2, \length{600}}{Маршрут уходит вверх-налево, в
широкий, хорошо выраженный ледовый кулуар (снизу перемычку не видно)}

\routeKey{R2-R3, \length{100}}{Выраженный ключ отсутствует. Но изредка
перед выходом на
\geopoint{42.510699}{74.5576562}{}{перемычку}{перемычка} последние
полторы веревки покрыты фирном, который безопасно можно пройти только
вырубая ступени, либо вертикально вверх по правым (по ходу движения)
скалам 15 метров лазания IV-V к.с.}

После подъема сразу оказываемся на ледовых подушках коронского
ледника. По ним спуск до \geoKoronaHut[Коронской хижины] по закрытому
леднику. Настоятельно рекомендуется движение в связках, есть трещины!
Развязаться можно перед самым последним спуском.
