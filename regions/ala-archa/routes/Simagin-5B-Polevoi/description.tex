От \geoKoronaHut{коронской Хижины} подход час-полтора.

\routeSection{R0-R1, \length{30}}{От берга под скалы пол верёвки обычного льда.}

\routeSection{R1-R2, \length{120}, \Gls{uiaaV}}{Аутентичный вариант обходит нижнюю скальную лапу слева. Там лазание до пятёрки, но разрушено.
Я лез в лоб по лапе. Лазание пять плюс и ИТО А1. В тапках вторую половину верёвки можно лезть. Верёвка \length{60} заканчивается перед
провалом в ребре, куда выходит аутентика.}

\routeSection{R2-R3, \length{70}}{Через провал вылазим на гребешок и пешком подходим к началу третьей верёвки. Третья верёвка начинается по
углу справа. Ориентиры родная морква и клин. После полочки с родным крюком продолжаем лезть по правой части и потом выходим на кант. По
канту вылазим на хорошую полку, конец третьей верёвки.}

\routeSection{R3-R4, \length{60}, \Gls{uiaaV}}{Четвёртая верёвка идёт справа в обход скальных башен с глушняковым рельефом. Лазание
пятёрочное плюс немного ИТО А1.}

\routeSection{R4-R5, \length{60}}{Пятая верёвка выходит в провал на ребре и продолжается по крутой стенке на левой части ребра. Куча родного
железа, морковки и крючья. Можно закончить верёвку на небольшой полочке перед уходом направо в откол. Можно в отколе, на родной петле.}

\routeSection{R5-R6, \length{60}}{Шестая верёвка из откола выходит налево на кант и по простым, но сильно заснеженным скалам идёт по канту до
стенки на хорошей полке с родным крюком откуда уходит направо по ходу ребра. Финиш шестой верёвки под началом крутой стенки слева по ходу
ребра.}

\routeKey{R6-R7, \length{60}, \Gls{uiaaV}}{Седьмая верёвка начинается с крутой стенки с кучей родного железа и потом переходит в угол,
залитый льдом. Лазание пятёрка и ИТО А1. На середине седьмой верёвки развилка.

Логично смотрится выход вверх направо на кант ребра, но там ожидает самый трудный кусок. Якорная стенка с ИТО А3. Аутентика уходит налево в
обход этой башенки. Обход виден не явно и надо чуть пролезть налево, чтобы его разглядеть. Там лазание пятёрка, но тоже разруха и снегА.}

\routeSection{R7-R8, \length{60}}{Восьмая верёвка идёт справа налево по крутой стенке и выходит на кант с куском неприятных снежных ножей.
Лазание плюс простое ИТО.}

\routeSection{R8-R9, \length{60}}{Девятая верёвка идёт по канту. Лазание плюс простое ИТО.}

\routeSection{R9-R10, \length{60}}{Десятая верёвка по канту подходит под башню с широким камином посередине. Лезется слева в обход башни и
по крутому снежно-скальному склону выводит на ребро. Справа на балде будет родная дюльферная петля из двух синих верёвок.}

\routeSection{R10-R11, \length{60}}{Одиннадцатая верёвка идёт по канту лазание плюс простое ИТО и выходит на гребень, где соединяется с
маршрутом 4Б.}

\routeSection{R11-R12, \length{120}}{Дальше по гребню две верёвки троечного лазания выводят на \geoPeakSimagina{вершину}.}

Спускаться удобнее по пути подъёма по родным петлям. На гребне закладка с петлёй. По гребню есть две родных петли под коротыши. С петли на
балде есть петли под половину шестидесятки до льда. При спуске по льду можно чередовать проушины и родные петли на скалах.

Резюме. Крепкая классическая альпинистская пятёрка. Отличный рельеф для страховки везде, кроме ключевой стенки, которую можно не лезть.
Единственная проблема~--- заснеженность. Станции все на нормальных полках. С полками под ночёвку тоже нет проблем. Железо идёт всё.

Маршрут практически абсолютно камнебезопасный. Сыпануть можно только на дюльферах в сторону ледового кулуара.
