От хижины \geoLighthouse{} подняться почти до верха
\geoKrokodil[морены «Крокодил»], ориентир~--- слева характерная
плоскость \geoTelevizor{}.

\routeSection{R0-R1, \length{200}, \Gls{uiaaI}}{Траверсировать склон
под ней до \geopoint{42.5244177762747}{74.53894497472571}{3790}{петли
в стене}{стартовая петля}. На \length{10} дальше, в выраженном
кулуаре, начало технической части маршрута.}

\routeSection{R1-R2, \length{80}, \Gls{uiaaII}, \Gls{uiaaIII}}{Первая
веревка~--- движение по кулуару вверх. Из него могут лететь камни,
поэтому страховать лучше от петли. От
\geopoint{42.5238249581}{74.53924046033}{3835}{второй станции}{вторая
станция} очередной внутренний угол пройти по правой стороне, до
\geopoint{42.52354030139}{74.53934774869}{3862}{наклонной
полки}{наклонная плока}. Заложиться возможно не везде.}

\routeSection{R2-R3, \length{50}, \Gls{uiaaII}}{Бараньи лбы по правую
сторону гребня, до
\geopoint{42.523310005249435}{74.53955561988748}{3888}{четвертой
станции}{четвертая станция}. Летом проходятся без проблем, зимой
возможно придется слегка пострадать. Первые 10 метров мало рельефа для
организации страховки, щели глухие.}

\routeSection{R3-R4, \length{80}, до \Gls{uiaaIII}}{Обход гребневого
жандарма по наклонным полкам, \length{40} \Gls{uiaaII}, после чего
\length{10} \Gls{uiaaIII} внутреннего угла до гребня. Здесь
\geopoint{42.523018427377856}{74.53948588245254}{3900}{пятая
станция}{}.

Движение по гребню, \length{30}, \Gls{uiaaII}. Страховка через рельеф.}

\routeSection{R4-R5, \length{50}, \Gls{uiaaIII}}{От
\geopoint{42.5226952191257}{74.53958646529902}{3920}{шестой
станции}{шестая станция} внутренний угол, много трещин. Страховка:
закладные элементы.}

\routeSection{R5-вершина, \length{40}, \Gls{uiaaII}}{Движение по
простому гребню с развальней до \geoPeakRacek{}.}

Спуск с вершины на другую сторону, по маршруту 2А.
