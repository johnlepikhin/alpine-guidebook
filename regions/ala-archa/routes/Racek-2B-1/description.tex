От хижины \geoLighthouse{} подняться почти до верха
\geoKrokodil[морены «Крокодил»], ориентир~--- слева характерная
плоскость \geoTelevizor{}.

\routeSection{R0-R1, \length{200}}{Траверсировать склон под ней до
петли в стене. На 10 метров дальше, в выраженном кулуаре, начало
технической части маршрута.}

\routeSection{R1-R2, \length{60}}{Первая веревка~--- движение по
кулуару вверх. Из него могут лететь камни, поэтому страховать лучше от
петли.}

\routeSection{R2-R3, \length{50}}{Вторая верёвка~--- выветренные скалы
с покатым рельефом. Летом проходятся без проблем, зимой возможно
придется слегка пострадать. Первые 10 метров мало рельефа для
организации страховки, щели глухие.}

\routeSection{R3-вершина, \length{160}}{Далее ещё примерно три веревки
простого рельефа, в основном можно идти одновременно или на
скользящем. Где-то по пути есть наклонная плита, по которой надо
аккуратно траверсировать метров пять.}

Спуск с \geoPeakRacek[вершины] на другую сторону, по маршруту 2А.
