От хижины \geoLighthouse{} подняться почти до верха
\geoKrokodil[морены «Крокодил»], ориентир~--- слева характерная
плоскость \geoTelevizor{}.

\routeSection{R0-R1, \length{200}, \Gls{uiaaI}}{Траверсировать склон
под ней до петли в стене. На \length{10} дальше, в выраженном кулуаре,
начало технической части маршрута.}

\routeSection{R1-R2, \length{80}, \Gls{uiaaII}, \Gls{uiaaIII}}{Первая
веревка~--- движение по кулуару вверх. Из него могут лететь камни,
поэтому страховать лучше от петли. Далее очередной внутренний угол
пройти по правой стороне, до наклонной полки. Заложиться возможно не
везде.}

\routeSection{R2-R3, \length{50}, \Gls{uiaaII}}{Бараньи лбы по правую
сторону гребня. Летом проходятся без проблем, зимой возможно придется
слегка пострадать. Первые 10 метров мало рельефа для организации
страховки, щели глухие.}

\routeSection{R3-R4, \length{200}, до \Gls{uiaaIII}}{Обход
гребневого жандарма по наклонным полкам, \length{40} \Gls{uiaaII},
после чего \length{10} \Gls{uiaaIII} внутреннего угла до гребня.

Движение по гребню, \length{30}, \Gls{uiaaII}. Страховка через рельеф.}

\routeSection{R4-R5, \length{50}, \Gls{uiaaIII}}{Внутренний угол,
много трещин. Страховка: закладные элементы.}

\routeSection{R5-R6, \length{10}, \Gls{uiaaI}}{Движение по простому
гребню с развальней.}

\routeSection{R6-вершина, \length{45}, \Gls{uiaaII}}{Предвершинный
гребень, разрушенные скалы.}

Спуск с \geoPeakRacek[вершины] на другую сторону, по маршруту 2А.
