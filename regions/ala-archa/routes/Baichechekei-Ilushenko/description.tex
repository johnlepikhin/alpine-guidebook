Подход

От хижины Light house тропа уходит вверх на морены ледника Учитель. По
ним двигаться ближе к левому борту (под ходу движения) около часа до
появления на левом борту довольно высоких скал. Перед ними начинается
подъём вверх. До входа в ущелье подниматься около 30 могут, далее по
ущелью ещё около 20 минут, до выраженного ледового кулуара справа под
ходу движения. Это маршрут.

Рекомендую развесить железо не доходя одну верёвку до льда т.к.
непосредственно перед началом маршрута даже вытоптать полочку в снегу
не всегда получается.

Сужение воронки проходить ближе к правому борту, три веревки. По
кулуару до ключа еще одна верёвка. Ключ - ледовая стенка до 85
градусов, 10-15 метров. После ключа до выхода из кулуара ещё 2.5
веревки относительно крутого льда. Воронку лезть 3 веревки пологого
льда. Строго вверх, не поддаваться соблазнам свернуть на скалы или
уйти за перегиб где-нибудь пониже. Ориентир - вершинка наверху, надо
выйти за перегиб чуть правее неё. В 20 метрах от перегиба начинается
курумник. Верхняя станция на какой-нибудь балде, их там много.

От полки правее вершины уходит спусковая тропа. По ней без потери
высоты двигаться около 200 метров, до тех пор пока внизу не станет
видно язык ледника Учитель. Спуск к нему по кулуару. Далее спуск вдоль
него до сворота на томские стоянки, пройти через них (соблазну уйти
сразу на морены не поддаваться!), далее спуск на морены и возвращение
домой.
