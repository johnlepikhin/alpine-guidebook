\routeSection{R0-R1, \length{1200}}{От хижины \geoLighthouse{} тропа
уходит вверх на морены ледника Учитель. По ним двигаться ближе к
левому борту (под ходу движения) около часа до появления на левом
борту \geopoint{42.5280865}{74.544038}{}{довольно высоких скал}.}

\routeSection{R1-R2, \length{600}}{Немного недоходя скал начинается
подъём вверх. До входа в ущелье подниматься около 30 минут, далее по
ущелью ещё около 20 минут, до выраженного
\geopoint{42.53048430863}{74.54980015218}{}{ледового кулуара} справа под
ходу движения. Это начало технической части маршрута.}

Рекомендую развесить железо не доходя одну верёвку до льда т.к.
непосредственно перед началом маршрута даже вытоптать полочку в снегу
не всегда получается.

\routeSection{R2-R3, \length{180}}{Нижнюю воронку лучше проходить
ближе к правому борту, три веревки. По кулуару до ключа еще одна
верёвка. Уклон льда около 55 градусов.}

\routeSection{R3-R4, \length{150}}{\routeKey{Ключ~--- ледовая стенка
до 85 градусов, 10-15 метров.} После ключа до выхода из кулуара ещё
2.5 веревки льда 60-65 градусов.}

\routeSection{R4-R5}{Воронку лезть 3 веревки пологого льда. Строго
вверх, не поддаваться соблазнам свернуть на скалы или уйти за перегиб
где-нибудь пониже. Ориентир~--- вершинка наверху, надо выйти за
перегиб \geopoint{42.52940962673}{74.55377576527}{}{чуть правее неё}. В
20 метрах от перегиба начинается курумник. Верхняя станция на
какой-нибудь балде, их там много.}

От полки правее вершины уходит спусковая тропа. По ней без потери
высоты двигаться около 200 метров, до тех пор пока внизу не станет
видно \geopoint{42.5234937}{74.5552216}{}{язык ледника Учитель}. Спуск к
нему \geopoint{42.5278164}{74.5549577}{}{по кулуару}. Далее спуск вдоль
него до сворота на \geoCampTomsk{}, пройти через них. Соблазну уйти
сразу на морены не поддаваться! Далее, спуск на морены и возвращение
домой.
