От хижины \geoLighthouse{} подход 1.5-2.5 часа.

\photo{\routeDiskPath}{1}{\routeTitle~--- нитка маршрута}

На подходе важно вовремя свернуть направо в направлении языка ледника Учитель и не уйти выше под Байчечекей. На свороте с натоптанной тропы
под Байчечекей есть два больших, двойных тура. Также ориентиром может служить первая скальная гряда над Учительским ледником. Она находится
немного выше тропы.

Дальше, в зависимости от состояния снега можно либо выбраться на срединную морену и идти по ней, либо спуститься на ледник Учитель и
продолжить путь по нему.

Если снега совсем много, есть третий вариант через \geoUchitelHut{Учительскую хижину}. Тропа проходит по левой морене Учительского ледника,
есть туры. Дойти до хижины и уже от неё пересечь ледник и выйти на срединную морену.

По срединной морене спуститься на ледник в направлении первой башни Короны. После подъёма на подушку под стеной первой Башни надо быть
внимательным, есть опасные трещины!

Есть два варианта начала маршрута. Первый вариант на верёвку длиннее, но под вторым может оказаться сильно разорванный бергшрунд, поэтому
вариант начала надо выбирать на месте, исходя из состояния склона.

При подходе под склон надо обратить внимание на состояние снега, возможен сход доски.

\routeSection{R0-R1}{Варианты:
\begin{itemize}
\item Справа от скального острова под северной стеной первой башни Короны.
\item Слева от скального острова. Склон в начале может быть камнеопасным в тёплую или ветреную погоду. Слева с осыпей иногда летят приличные
  камни.
\end{itemize}}

\routeSection{R1-R2}{ По ледовому склону средней крутизны маршрут подходит к узкому кулуару между первой башней Короны и
жандармом на восточном гребне. По кулуару выход на перемычку между первой башней и жандармом. Общая длина ледовой части по первому
варианту~--- 14 верёвок.}

С перемычки начинается скальная часть маршрута.

\routeSection{R2-R3, \length{250}, \Gls{uiaaIV}}{ Маршрут проходит слева от канта гребня по разрушенным и в некоторых местах нависающим
скалам и выходит на кант гребня недалеко от вершинной башни. Лазание до четвёрочного. Общая протяжённость~---5 верёвок.

С плеча над спусковым кулуаром по несложным скалам маршрут выходит на \geoPeakKoronaOne{вершину}.}

Спуск до Коронского ледника дюльфером по пути подъёма до плеча, а затем по ледовому кулуару. Практически везде есть петли на скалах. При
спуске в кулуар необходимо быть аккуратным, чтобы не засыпать друг друга камнями с осыпной полки над кулуаром. Далее путь спуска проходит по
Коронскому леднику до ледника Аксай и по нему до стоянок Рацека. На Коронском леднике необходимо быть внимательными на
верхней подушке и внизу, в районе чёрного жандарма есть опасные трещины.

Классическая комбинированная четвёрка. Маршрут длинный и физически достаточно тяжёлый. Надо быть готовыми к спуску по темноте.
