\routeSection{R0-R1, \length{700}}{За кухней хижины \geoLighthouse{}
начинается тропа сразу вверх, по ней подняться на 50 метров выше
последних скал слева.}

\routeSection{R1-R2, \length{550}}{Подняться на морену налево и
продолжить движение в том же направлении. Есть немного туров, но в
темноте их обычно не видно. Пересечь первую долинку, здесь лучше
двигаться по дуге без потери высоты. Пересечь вторую, довольно
обширную долину. На курумник справа смысла лезть нет, лучше обойти на
10 метров ниже. Подняться на гребешок 20 метров, примерно здесь
находится большой \geoStartOfKoshevoy[тур]. Но тропу в цирк
\geoPeakKoshevoy{} с этого гребня обычно видно и без тура.}

\routeSection{R2-R3, \length{500}}{Траверсировать склон по тропе с
постепенным спуском в цирк.}

\routeSection{R3-R4, \length{600}, \Gls{uiaaI}}{Маршрут начинается с
подъема по сыпухе в самый правый по ходу движения кулуар.}


\routeSection{R4-R5, \length{300}, \Gls{uiaaII}}{Перевалить через
седловину и траверсировать склон влево по ходу, на 30-50 метров ниже
гребня, около 300 метров. Двигаться до характерной пробки (застрявший
в щели камень диаметром 80 см).}

\routeSection{R5-R6, \length{70}, \Gls{uiaaIII}}{Пробку облезть,
приспуститься 3 метра в кулуар. От этого места по осыпному кулуару
строго вверх, сложность II. Обычно 50-метровой веревки хватает ровно
до гребня. Там хорошая площадка, можно на закладках организовать
станцию. Отсюда до \geoPeakKoshevoy[вершины] 15 метров пешком без
страховки.}

Спуск по пути подъема.
