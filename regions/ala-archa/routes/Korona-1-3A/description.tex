
\routeSection{R0-R1, \length{2000}}{От \geoKoronaHut[Коронской хижины]
  подняться на самую верхнюю ледовую подушку. \geoPeakKoronaOne[Башня]
  имеет справа (по ходу движения) жандарм из тёмных пород. У основаня
  башни и жандарма ледник разрывает неширокий бергшрунд.}

Далее есть два варианта маршрута:

\begin{itemize}
\item Более сложный вариант. \routeSection{R1-R2, \length{100},
    \Gls{uiaaV}}{Обойти жандарм справа, по кулуару вылезти на
    перемычку. Зимой кулуар во льду.} \routeKey{R2-R3, \length{20}}{С
    перемычки 5 метров пятерочного лазания на полку на жандарме. Далее
    по узкому ступенчатому гребню жандарма пролезть 15 метров до
    дюльфера. Со страховкой на гребне жандарма непросто. Дюльфер 8
    метров на перемычку перед башней, далее варианты соединяются (см.
    ниже).}

\item Более простой вариант. \routeSection{R1-R3,
    \length{100}}{Подняться слева от жандарма до перемычки по кулуару
    (зимой лёд).}
\end{itemize}

Далее линии вариантов объединяются:

\routeKey{R3-R4, \length{6}, \Gls{uiaaIV}}{Ключ второго варианта и
  продолжение первого варианта. 5-7 метров скал, сложность V, 80
  градусов. Висит много петель и старых веревок, использовать с
  осторожностью.}

\routeSection{R4-R5, \length{170}, \Gls{uiaaIII}}{Три-четыре веревки
  скал, ориентироваться в сторону вершины. Наклонные плиты, развальня.
  Страховка через рельеф, петли, частично закладные элементы.}

Спуск по 2-му варианту подъёма.
