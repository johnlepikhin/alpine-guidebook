От хижины \geoLighthouse{} подняться по \geoKrokodil[морене
«Крокодил»] до самого верха. Далее пересечение ледника к заснеженному
кулуару между п.~Теке-Тор и гребнем \geoPeakBoks{}. Ледник идти в
связках, есть трещины.

\routeSection{R0-R1, \length{700}}{Кулуар выходит на \geoPassTekeTor{} и
проходится пешком, обычно много тропежки. Наверху кулуар разбивается
на три части. Наиболее безопасно идти по центральному.}

\routeSection{R1-R2, \length{400}, \Gls{uiaaIII}}{Техническая часть
начинается на седловине. Жандарм обойти слева по ходу движения. Далее
около двух веревок лазания по развальне в слабо выраженном широком
кулуаре, движение в основном одновременное со страховкой через рельеф.
Пройти около 200 метров по гребню почти без набора высоты, упираемся в
хорошо выраженный угол.}

\routeSection{R2-R3, \length{7}, \Gls{uiaaV}}{\routeKey{Угол~--- 7 метров
пятерочного лазания. Это ключ маршрута. Страховка через 1-2 френда
примерно 4 размера в щели. Зимой проходится довольно напряженно, есть
возможность пройти на 10 метров левее, через «инструкторский» ход (4
метра лазания IV, с облазом большой балды наверху).}}

\routeSection{R3-R4, \length{210}, \Gls{uiaaII}}{Простое движение по
гребню, заканчиваются дюльфером на левую сторону гребня. Плиты 4 метра
75 градусов, свободным лазанием спуск затруднён. Есть петля.}

\routeSection{R4-R5, \length{230}, \Gls{uiaaII}}{После дюльфера
выходим снова на правый борт гребня, около 150 метров, движение
одновременное. Спуск на \geopoint{42.5211174}{74.5240895}{перемычку},
80 метров, угол около 45 градусов.}

\routeSection{R5-R6, \length{90}, \Gls{uiaaI}}{От перемычки движение
пешком справа от жандарма, с выходом на очередную перемычку. С этого
места наконец-то открывается вид на вершину, до которой 30 метров
пешком по полностью безопасному гребню с набором высоты 10 метров.}

Спуск по маршруту 1Б. Для этого надо пройти по скальному гребешку
вдоль ледника около 20 метров. На лёд спускаться не надо! Там в
кулуаре \geopoint{42.52241638343}{74.52510528836}{начинается тропа},
верхие 10-20 метров которой отлично видны при отсутствии снега. Спуск
по кулуару в целом без проблем, но надо контролировать камни, можно
друг друга засыпать. Пересечь ледник обратно к Крокодилу, так же в
связках.
