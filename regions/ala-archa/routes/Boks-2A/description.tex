От хижины \geoLighthouse{} подняться по \geoKrokodil[морене
«Крокодил»] до самого верха. Далее пересечение ледника к заснеженному
кулуару между п.~Теке-Тор и гребнем \geoPeakBoks{}. Ледник идти в
связках, есть трещины. Кулуар выходит на \geoPassTekeTor{} и
проходится пешком, обычно много тропежки. Наверху кулуар разбивается
на три части. Наиболее безопасно идти по центральному.

Маршрут начинается на седловине. Жандарм обойти слева по ходу
движения. Далее около двух веревок лазания по развальне в слабо
выраженном широком кулуаре, сложность II-III, движение одновременное
со страховкой через рельеф. Пройти около 200 метров по гребню почти
без набора высоты, упираемся в хорошо выраженный угол: \routeKey{7
метров пятерочного лазания. Это ключ маршрута. Страховка через френды
примерно 5 размера в щели. Зимой проходится довольно напряженно, есть
возможность пройти на 10 метров левее, через «инструкторский» ход (4
метра лазания IV, с облазом большой балды наверху).}

Далее ещё 200 метров по гребню, которые заканчиваются дюльфером на
левую сторону гребня. Плиты 4 метра 75 градусов, свободным лазанием
спуск затреднён. Есть петля. После дюльфера выходим снова на правый
борт гребня, около 150 метров, движение одновременное. Спуск на
\geopoint{42.5211174}{74.5240895}{перемычку}, 80 метров, угол около 45
градусов.

От перемычки движение пешком справа от жандарма, с выходом на
очередную перемычку. С этого места наконец-то открывается вид на
вершину, до которой 30 метров пешком по полностью безопасному гребню с
набором высоты 10 метров.

Спуск по маршруту 1Б. Для этого надо пройти по скальному гребешку
вдоль ледника около 20 метров. На лёд спускаться не надо! Там в
кулуаре \geopoint{42.52241638343}{74.52510528836}{начинается тропа},
верхие 10-20 метров которой отлично видны при отсутствии снега. Спуск
по кулуару в целом без проблем, но надо контролировать камни, можно
друг друга засыпать. Пересечь ледник обратно к Крокодилу, так же в
связках.
