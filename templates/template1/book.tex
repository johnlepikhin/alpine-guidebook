
\documentclass[11pt,fleqn]{report} % Default font size and left-justified equations

\usepackage[T1,T2A]{fontenc}
\usepackage[english,russian]{babel}
\usepackage[utf8]{inputenc}
\usepackage{indentfirst}
\usepackage{graphicx}
\usepackage{color}
\usepackage{xcolor}
\usepackage{numprint}
\usepackage{mdframed}
\PassOptionsToPackage{hyphens}{url}\usepackage{hyperref}

\usepackage{alpineroutes}

\renewcommand{\length}[1]{длина #1~м}

\renewcommand{\geopoint}[3]{ \href{https://www.google.com/maps?q=#1,#2}{#3
    \textcolor{gray}{(координаты \nprounddigits{6}\numprint{#1}, \numprint{#2})}}}

\newmdenv[]{MdEnv}

\renewcommand{\routeSection}[2]{
  \begin{MdEnv}[topline=false,bottomline=false,rightline=false,linewidth=2pt]
    \textbf{
      Участок #1:
    } #2
  \end{MdEnv}
}

\begin{document}

\chapter{Учебные маршруты Ала-Арчи}

\routeListEasyAlaArcha

\chapter{Маршруты Ала-Арчи средней сложности}

\routeListMiddleAlaArcha

\chapter{О проекте}

\newcommand{\gitRepo}[0]{\href{https://github.com/johnlepikhin/alpine-guidebook}{GitHub}}

Пока что я встречал два вида описаний маршрутов:
\begin{itemize}
\item Гайдбук типа "5000 игр на одном CD". Качество описаний и схем
  таково, что авторов хочется обнять и плакать.
\item 20 страниц подробного описания маршрута 3А категории сложности.
\end{itemize}

Всё это крайне неуодбно, когда в толпе и морозе альплагеря надо
составить маршрутный лист, а ты уже потратил пару часов на поиск хотя
бы вот таких описаний.

Нет цели заработать денег~--- проект полностью открытый~--- и нет цели
играть в сурового первопроходца и писать 20-страничные портянки с
описанием каждого камня. Зато есть желание сделать \textit{удобное}
описание популярных, ходимых маршрутов.

\section{Как можно помочь}

Нужны авторские описания и схемы. Описания в формате \LaTeX, но с
радостью приму и обычный doc-файл, или даже словестные описания
неточностей :-) Схемы рисую в SVG, и хотелось бы сохранить эту
традицию для простоты совместной правки.

Все исходные тексты доступны на \gitRepo. Можно создавать merge
requests, можно присылать тексты/комментарии на
\href{mailto:johnlepikhin@gmail.com}{johnlepikhin@gmail.com} или
связаться со мной любым удобным способом.

\section{Права на использование}

Пока не разобрался с лицензией и не выбрал наиболее подходящую
свободную лицензию, прямо запрещаю использование размещенных здесь
материалов в коммерческих или каких-либо иных целях. Разрешается
публикация ссылок на авторский сайт проекта и его репозиторий,
расположенный на \gitRepo.

\end{document}
